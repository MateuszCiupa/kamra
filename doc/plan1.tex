\documentclass{article}
\usepackage[T1]{fontenc}
\usepackage[polish]{babel}
\usepackage[utf8]{inputenc}
\usepackage[none]{hyphenat}

\title{Kamera wirtualna (część 1.)}
\author{Mateusz Ciupa 291062}
\date{\today}

\begin{document}
\maketitle

\section{Wstęp}
Celem pierwszej części projektu jest zaimplementowanie kamery wirtualnej w dowolnym języku.
Kamera powinna umożliwiać dokonywania translacji (lewo, prawo, góra i dół), rotacji wokół
trzech osi oraz operacji zoom.

\section{Plan pracy}

\begin{enumerate}
    \item Ćwiczenie zostanie wykonane przy użyciu języka \textit{JavaScript} do obsługi 
    operacji matematycznych oraz elementu \textit{Canvas} języka HTML do rysowania obiektów.
    \item Pierwszym elementem będzie stworzenie krawędzi (dwóch punktów w układzie
    współrzędnych kamery), z których będą składać się wyświetlane obiekty.
    \item Punkty będą podlegać kolejnym przemianom (mnożenie punktów przez macierze):
    \begin{itemize}
        \item \textit{perspective projection},
        \item \textit{orthographic transform} (aby otrzymać \textit{canonical view volume}),
        \item \textit{viewport transformation} (dopasowanie do rozmiarów wyświetlanego obrazu).
    \end{itemize}
    \item Translacja, rotacja oraz zoom kamery będą dokonywane poprzez mnożenie punktów 
    tworzących obiekty (układu współrzędnych kamery) przez odpowiednie macierze oraz wykonanie
    operacji z wcześniejszego punktu. Operacje kamery będą obsługiwane przy pomocy klawiatury.
\end{enumerate}
 
\end{document}
